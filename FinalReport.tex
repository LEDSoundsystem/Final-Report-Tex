\documentclass[11pt]{article}


\usepackage{amsmath}
\usepackage{fancyhdr}
\usepackage{enumitem}
\usepackage{siunitx}
\usepackage{graphicx}
\usepackage{listings}
\usepackage{booktabs}
\usepackage{url}
\usepackage{cite}
\usepackage{tikz}
\usetikzlibrary{shapes.geometric, arrows}
\usepackage[section]{placeins}

\usepackage[letterpaper, portrait, margin=1in]{geometry}
\DeclareGraphicsExtensions{.pdf,.png,.jpg}
\graphicspath{ {images/} }

\newcommand{\sinum}[1]{\num[scientific-notation=true]{#1}}

\rhead[L]{FInal Report \thepage}


\begin{document}

\begin{titlepage}
    \centering
    \title{LED Soundsystem: Quantifying the Effects of Music And Light on Human Listeners \\}
\date{5-10-16}
\author{Evan Nichols, Cole Jurden, Ian Nelson}
\end{titlepage}

\pagenumbering{gobble}
\maketitle
\newpage
\pagenumbering{arabic}
\tableofcontents
\newpage

%%%% INTRODUCTION %%%%%
\section{Introduction}
Americans spend an average of 4 hours and 5 minutes listening to music each day.\cite{earshare} Music can have both positive and negative effects on the human body and mind. Athletes listen to pump-up music to prepare for games. Yoga instructors play Kirtan or Tibetan chants to enhance the focus of their participants. Music therapy programs seek to address physical, emotional, cognitive, and social needs of individuals. Different sounds provide different sensations to a human listener. 

The same type of effects can be said for light: soft candle lighting sets an intimate, relaxing environment, while the harsh fluorescent lighting of a classroom at 8:00 a.m. can be a rude awakening for bleary-eyed students. Red lighting puts us in a heightened state of alertness or awareness, while blue tends to relax us.

We experience these two mediums so constantly that our physiological reactions to their stimulus are very much subconscious. Our project, LED Soundsystem, will gauge user's physiological response to different music and light combinations. In particular, we will analyze different lighting environments and their effects on listener's heart rate. These findings will be paired with key acoustic attributes of each song -- danceability, energy and valance -- to see which attribute has the most impact on changes in heart rate. Additionally, the system will runs the above tests under three separate lighting environments to measure how much it affects physiological response as well.

%%%% PROBLEM STATEMENT %%%%%
\section{Problem Statement}

There is relatively  ubiquitous components of human life -- music and light 
There are many potential benefits to LED Soundsystem. For one, it gives users a firsthand look into their physical responses to light and music. They will be able to identify which music and light combinations provide positive reactions, and which ones do not. This increased awareness could potentially improve the overall well-being of users. On a broader scale, aggregated data from this application could be incredibly valuable to Music Therapy researchers seeking to find the most therapeutic combinations of music and light.

%%%%% SYSTEM DESIGN %%%%%%
\section{System Design}

\begin{center}
\begin{figure}[ht]
\includegraphics[scale=0.5]{blockdiagram}
\caption{System Design block diagram.}
\end{figure}
\end{center}

The final system design consists of 4 primary components: the iOS application, the Apple Watch application, the NodeJS/Express Raspberry Pi Server, and the Phillips Hue\textsuperscript{TM} LED lights. Each component and its purpose are described briefly below.

\begin{description}
 	\item[iOS Application] \hfill \\
The iOS Application handles authenticating the user's Spotify account (to allow access to their API) and allows users to select a song for which to take heart rate measurements. The application handles streaming the song, beginning a workout (which is required to gain access to heart rate data) and sending relevant song data and heart rate information to the Raspberry Pi server.
	\item[Apple Watch Application] \hfill \\
The Apple Watch application's only purpose is to accept the "begin workout" request from the iOS app and stream heart rate data to the phone while the song plays.
	\item[Raspberry Pi NodeJS/Express Server] \hfill \\
The server handles storing data from the iOS app and creating output graphs for the heart rate and song attributes. It uses an array to store song information, and an array within each song to keep track of heart rate as data points are received via POST calls. When a song is complete, the information is formatted and sent to Plotly to generate graphs depicting heart rate vs. time and acoustic attributes vs. percentage change in heart rate. These graphs show us which part of the songs and acoustic attributes have the greatest physiological effects.
	\item[Phillips Hue\textsuperscript{TM} LED Lights] \hfill \\
The lights constitute our final experimental variable. We ran the applications above using three different lighting atmospheres -- strong dark blue, light yellow and a plain white light as the control. With all other factors held constant, this component allows us to gauge how much lights effect heart rate. The light color is taken into account in the Plotly graphs.  
\end{description}


%%%% RESULTS AND ANALYSIS %%%%%%
\section{Results and Analysis}
Need:
Regression graphs
Heart rate graphs
Some description/analysis of results 

%%%% RELATED WORK %%%%%
\section{Related Work}
A UC Berkeley study revealed that connections between music and color was strongly influenced by emotion. \cite{Palmer} Participants were asked to match 18 different selections of classical music to a color. Faster music in a major mode was consistently matched with lighter, yellower color choices while slower music in a minor mode produced darker, more saturated blue colors. Participants from both the United States and Mexico were included in the study to take into account potential "cultural dependencies;" however, the patterns and color choices were consistent across both cultures.

 Listening to classical and relaxing music after exposure to stressors can reduce anger and anxiety, and stimulate the sympathetic nervous system.\cite{musicstress} Similarly, light has distinct effects on mood and emotional response. Ambient brightness triggers the intensity of human's affective response, while reducing lights can reduce the emotionality of everyday decisions.\cite{light} 

%%REWRITE BELOW%%
There are many studies regarding how music and light affect mood independently. In the case of music, studies directly linking certain genres to specific moods are abundant. Specifically, one paper analyzed the affect of four genres of music on mood. The four genres tested were grunge rock, classical, new age, and designer. The study tested different age groups by conducting a control survey featuring mood indicators and measured changes in results after exposure to different music. The study observed that every genre did affect mood in varying intensities \cite {mccraty}. Another article attempts to suggest music to a user based on mood, quantifying the music into four ?mood? tags using qualities such as tempo. Though the sample size was not large enough to be truly meaningful, the results of the experiment show that mood can be inferred from inherent qualities in music \cite {feng}. 

Other research shows links between ambient brightness and humans' affective response. Bright sunny days have both positive and negative effects: they can fill the heart with joy, but they are also associated with heartbreak. \cite{kevan} Similarly, sunny days cause depression-prone people to become more depressed. Suicides peak in late spring and summer, and reach their lowest point during the winter months. Bright light tends to amplify a person's initial emotional response, whether that be positive or negative.\cite{xu} 


\newpage 
\addcontentsline{toc}{section}{References}
\bibliographystyle{plain}
\bibliography{mybib}

\end{document}
